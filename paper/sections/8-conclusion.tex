\section{Conclusion}

Grounded understanding presumes common world and contextual understanding of a given topic. To understand the impact of this contextual information on NLU systems we enhanced the model presented in \citep{monroe-2017-colors} with color representations encoded within convolutional neural networks such as ResNet and text descriptions represented as contextual word embeddings extracted from pretrained transformers like ELECTRA or XLNet. We conducted a series of 45 experiments using different model setups and fine-tuned those models along our defined hyperparameter space.

\par
While we were able to achieve state-of-the-art results using ResNet with Fourier transform and ELECTRA contextual word embeddings, we showed that more complex color and text representations don’t necessarily perform better on the colors dataset. Given the low complexity of the utterances in the color dataset, we find that these higher dimensional representations often inject noise to the system which leads to only minimal gains at the cost of greater computational expenses.

% \begin{itemize}
%   \item briefly summarize what paper did and why
%   \item articulate broader significance of work
%   \item should be short and on point
% \end{itemize}
