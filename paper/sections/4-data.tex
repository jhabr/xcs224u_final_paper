\section{Data}

We evaluate our models on the color dataset used in \citep{monroe-2017-colors}. This dataset was generated by playing reference games where the color descriptions were produced by human participants in the speaker role. On each round of the game, the speakers were presented with three color patches, one of which was selected to be target color. The speakers were instructed to communicate this information to the listeners who were asked to click on one of the colors to complete the task \citep{monroe-2017-colors}.

\par
The dataset was split to train and test set with 46,994 and 2,031 examples respectively. Every dataset comprises of examples containing three colors, of them being the target color and two distractors. Every example belogs to one of three conditions: 1) \emph{close}, where all three colors where similar, 2) \emph{split}, where one of the colors is close to the target colors and 3) \emph{far}, where all three colors are far appart in the color space \citep{monroe-2017-colors}. Table \ref{table:colors} shows one example from each condition.

\begin{table}[ht]
\centering
\renewcommand{\arraystretch}{1}
\begin{tabular}{|l|l|l|}

  \hline
  Colors & Condition & Utterance \\
  \hline

\end{tabular}
\caption[Colors]{Colors...}
\label{table:colors}
\end{table}


\begin{itemize}
  \item describe properties of data set
  \item actual examples, descriptions
  \item quantitative summaries (number of examples)
  \item motivation for dataset choice
\end{itemize}