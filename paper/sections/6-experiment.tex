\section{Experiment}

\subsection{Experiment Setup}
To address our hypotheses we use different model compositions which are based on our target architecture.

\textbf{Model for baseline}:
Our implementation of the encoder-decoder architecture model based on assignment 4 (based on a single layer of RNN’s and GRU’s, Fourier color representation).

\textbf{Models for hypothesis 1}:
For hypothesis 1, we are using Fourier transformation encoded context colors with pre-trained and contextual embedding from the following transformer models: \texttt{BERT}, \texttt{XLNet}, \texttt{Electra}, \texttt{Roberta}.

\textbf{Models for hypothesis 2}:
For hypothesis 2, we use ResNet for contextual color representation while keeping the same model embeddings.

\textbf{Models for hypothesis 3}:
For hypothesis 3, we build on hypothesis 2 and combine different hidden layers of the transformer models. Namely, we concatenate the last four hidden layers, sum the last four hidden layers and extract the second-to-last hidden layer.

\bigbreak
what to write:
\begin{itemize}
  \item explain how data and models work together for my experiments
  \item show which models were eveluated
  \item how models were trained
  \item data pre-processing
  \item which metrics were used
  \item what are experimental outcomes
\end{itemize}