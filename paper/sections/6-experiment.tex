\section{Experiment}

Table INSERT TABLE NUMBER summarizes our results. We list our results in order of number of parameters in the model. This is a combination of trainable and untrainable parameters, as the core architecture of our models is generally a flavor of a recurrent neural network sequence-to-sequence model. The core architectural differences are captured by the method we use for embedding words and colors, while our encoder-decoder architecture generally stays the same. We use GRU as a default cell for all experiments. We also used an LSTM cell for the experiments that have shown best model performance to evaluate how a LSTM affects the performance of the models in comparison with GRU.

\par
We experimented with four different hidden dimensions (hidden\_size) for the RNN cells. We used 50 (default for all experiments) , 100, 150, 250 to determine what hidden dimensions will have the most positive effect on the model performance. Those experiments were conducted on models with RoBERTa, BERT, XLNet or ELECTRA pre-trained word embeddings and Fourier transformed or ResNet concatenated with Fourier transformed colour embeddings. A subset of the train dataset (8,000 records) was used for those experiments. We limited the number of experiments due to time and budget limitations. Finally, we trained and tested the best performing models with a hidden dimension of 250 as our experiments showed that a higher dimension has a positive effect on the model output.

\par
For example, for our simplest model (ID \#1) we use a fourier transform to project the colors from a 3-dimensional vector to a 54-dimensional vector. We use 50-dimensional pre-trained GLoVe embeddings to embed each word in the corpus as a distributed representation. We use a sequence-to-sequence model with a GRU cell acting as a color encoder and text decoder.  For this baseline model (and all models) we set the hidden dimension of the GRU to 50. From \citep{dey-2017-gru} there are a total of \[3*(n^2 + n*m +n)\] where n is the hidden dimension of the GRU and m is the input dimension to the cell. So for our encoder our number of trainable weights becomes:

\[3*(50^2 + 50*54 +50) = 15,150\]

\par
and for the decoder our number of weights becomes:

\[3*(50^2 + 50*50 +50) = 15,750\]

\par
For the entire architecture this totals to 30,900 trainable parameters. We note that we freeze the word embeddings and color embeddings in all models.

\par
Now for a more complex model that incorporates both transformer pre-trained or contextual word embeddings and color embeddings generated by a convolutional encoder, our number of trainable parameters increases. For a model that uses 768-dimensional BERT embeddings and 512-dimensional ResNet encoded color embeddings our trainable parameters for the encoder becomes:

\[3*(50^2 + 50*512 +50) = 84,450\]

\par
and for the decoder our number of weights becomes:

\[3*(50^2 + 50*768 +50) = 122,850\]

\par
This results in a total of 207,300 trainable parameters. The number of trainable parameters significantly increases for the experiments that we used hidden dimension of 250. For the aforementioned example the number of trainable parameters used in the encoder becomes:

\[3*(250^2 + 250*512 +50) = 571,650\]

\par
And for the decoder the number of weights becomes:

\[3*(250^2 + 250*768 +50) = 763,650\]

\par
We note that while the more complex models have more power to capture complex interactions, the large models have nearly seven times the number of trainable parameters leading to concerns of building models with high variance. We will discuss the effects of model size and overfitting further in our Analysis section.

\par
For all models we stop training after the validation performance has increased for 10 iterations in a row at a tolerance of \(10^-5\). We use the listener accuracy and the BLEU score to evaluate model performance.

\par
Listener accuracy and BLEU score are used for evaluating the performance of the models. The former allows us to evaluate the ability of the trained model to construct accurate colour description utterances based on the colours input.

\[c^{*} = argmax_{c \in C} P_{s} (utterance | c)\]

\par
where \(P_{s}\) is the describer model and \(C\) is the set of all permutations of all three colors in the color context. We take \(c^{*}\) to be a correct prediction if it is one where the target is in the privileged final position \citep{potts-2020-colors}.

\par
The listener's accuracy assesses the ability of the model to communicate with itself and this can lead to generating utterances that are far from proper English. BLEU score is used to ensure that this situation doesn’t happen.

\par
We consider both listener accuracy and BLEU score, with equal performance when evaluating the models and therefore, a model is consider better performing if both scores are higher.

\bigbreak
[table goes here]



% \bigbreak
% what to write:
% \begin{itemize}
%   \item explain how data and models work together for my experiments
%   \item show which models were eveluated
%   \item how models were trained
%   \item data pre-processing
%   \item which metrics were used
%   \item what are experimental outcomes
% \end{itemize}